\subsection{Ни нечетная, ни четная функция}

Рассмотрим периодическую функцию с периодом $T = 2\pi$:

\begin{equation}
    f_4(t) = \sin(t)^3 - \cos(t)
\end{equation}

График этой функции приведен на рис.~\ref{fig:func_4}.

\begin{figure}[ht!]
    \centering
    \includegraphics[width=\textwidth]{media/plots/func_4.png}
    \caption{График функции $f_4(t)$}
    \label{fig:func_4}
\end{figure}

\subsubsection{Вычисление коэффициентов Фурье}
Найдем коэффициенты ряда Фурье для этой функции:

\begin{equation}
    a_n = \frac{1}{\pi}\int\limits_{-\pi}^{\pi} (\sin(t)^3 - \cos(t)) \cos(n t) dt 
\end{equation}

\begin{equation}
    b_n = \frac{1}{\pi}\int\limits_{-\pi}^{\pi} (\sin(t)^3 - \cos(t)) \cos(n t) dt 
\end{equation}

\begin{equation}
    c_n = \frac{1}{2\pi}\int\limits_{-\pi}^{\pi} (\sin(t)^3 - \cos(t)) e^{-int} dt
\end{equation}

\subsubsection{Вычисление коэффициентов Фурье с помощью программы}

\begin{lstlisting}[style=python_white, caption=Вычисление коэффициентов Фурье, label=lst:func_4]
func = np.vectorize(lambda x: np.sin(x) ** 3 - np.cos(x))
a, b = fourier(func, 0, 2 * np.pi, 3)
print_fourier_coefficients(a, b)
c = fourier_exp(func, 0, 2 * np.pi, 3)
print_fourier_exp_coefficients(c)
\end{lstlisting}

В результате выполнения программы (\ref{lst:func_4}) получим следующие значения (см. таблицу~\ref{tab:func_4}~и~\ref{tab:func_4_exp}).

% table with coefficients
\begin{table}[ht!]
    \centering
    \begin{tabular}{|c|c|c|}
        \hline
        $n$ & $a_n$ & $b_n$ \\
        \hline
        0 & -0.00020 & 0.00000\\
        1 & -1.00020 & 0.75000\\
        2 & -0.00020 & 0.00000\\
        3 & -0.00020 & -0.25000\\
        \hline
    \end{tabular}
    \caption{Коэффициенты Фурье для функции $f_4(t)$}
    \label{tab:func_4}
\end{table}

\begin{table}[ht!]
    \centering
    \begin{tabular}{|c|c|}
        \hline
        $n$ & $c_n$ \\
        \hline
        -3 & -0.00010-0.12500i \\
        -2 & -0.00010-0.00000i \\
        -1 & -0.50010+0.37500i \\
        0 & -0.00010+0.00000i \\
        1 & -0.50010-0.37500i \\
        2 & -0.00010+0.00000i \\
        3 & -0.00010+0.12500i \\
        \hline
    \end{tabular}
    \caption{Коэффициенты Фурье для функции $f_4(t)$ (комплексный случай)}
    \label{tab:func_4_exp}
\end{table}

\subsubsection{Построение графиков частичных сумм ряда Фурье}
В качество значений $N$ выберем $N = 1, 2, 3, 4, 5$. Для каждого значения $N$ вычислим частичную сумму ряда Фурье и построим график (см. рис.~\ref{fig:func_4_plot}~и~\ref{fig:func_4_plot_exp}).

\begin{lstlisting}[style=python_white, caption=Построение графиков частичных сумм ряда Фурье, label=lst:func_1_plot]
func = np.vectorize(lambda x: np.sin(x) ** 3 - np.cos(x))
calc_and_plot(func, 0, 2 * np.pi, [1, 2, 3, 4, 5], './media/plots/func_4')
calc_and_plot_exp(func, 0, 2 * np.pi, [1, 2, 3, 4, 5], './media/plots/func_4_exp')
\end{lstlisting}

% plot with partial sums
\begin{figure}[ht!]
    \centering
    \includegraphics[width=0.49\textwidth]{media/plots/func_4.png}
    \includegraphics[width=0.49\textwidth]{media/plots/func_4_N_1.png}
    \includegraphics[width=0.49\textwidth]{media/plots/func_4_N_2.png}
    \includegraphics[width=0.49\textwidth]{media/plots/func_4_N_3.png}
    \includegraphics[width=0.49\textwidth]{media/plots/func_4_N_4.png}
    \includegraphics[width=0.49\textwidth]{media/plots/func_4_N_5.png}
    \caption{График частичных сумм ряда Фурье для функции $f_4(t)$}
    \label{fig:func_4_plot}
\end{figure}

\begin{figure}[ht!]
    \centering
    \includegraphics[width=0.49\textwidth]{media/plots/func_4_exp.png}
    \includegraphics[width=0.49\textwidth]{media/plots/func_4_exp_N_1.png}
    \includegraphics[width=0.49\textwidth]{media/plots/func_4_exp_N_2.png}
    \includegraphics[width=0.49\textwidth]{media/plots/func_4_exp_N_3.png}
    \includegraphics[width=0.49\textwidth]{media/plots/func_4_exp_N_4.png}
    \includegraphics[width=0.49\textwidth]{media/plots/func_4_exp_N_5.png}
    \caption{График частичных сумм ряда Фурье для функции $f_4(t)$}
    \label{fig:func_4_plot_exp}
\end{figure}

Видим, что при увеличении $N$ график частичной суммы ряда Фурье приближается к исходной функции, и уже при $N = 3$ график частичной суммы ряда Фурье неотличим от исходной функции.
