\section{Исходный код}
\label{appendix:source}

\textbf{Исходный код можно найти по ссылке: \href{https://github.com/edelwiw/Frequency_methods}{GitHub}} 

\begin{lstlisting}[style=python_white, caption=Функция для вычисления скалярного произведения функций, label=lst:dot_product]
def dot_product(f, g, a, b):
    x = np.linspace(a, b, 10000)
    dx = x[1] - x[0]
    return np.dot(f(x), g(x)) * dx
\end{lstlisting}
Аргументы функции \texttt{dot\_product}: \texttt{f} и \texttt{g} -- функции, для которых вычисляется скалярное произведение, \texttt{a} и \texttt{b} -- границы интегрирования.
\newline

\begin{lstlisting}[style=python_white, caption=Функция для получения вспомогательных функций, label=lst:get_sincosmt]
def get_sincosmt(T):
    sinmt = lambda x: np.sin(2 * np.pi * x[0] / T * x[1])
    cosmt = lambda x: np.cos(2 * np.pi * x[0] / T * x[1])

    return sinmt, cosmt
\end{lstlisting}
Аргументы функции \texttt{get\_sincosmt}: \texttt{T} -- период функции.
\newline

\begin{lstlisting}[style=python_white, caption=Функция для получения вспомогательных функций, label=lst:get_expmt]
def get_expmt(T):
    return lambda x: np.e ** (1j * 2 * np.pi * x[0] / T * x[1])
\end{lstlisting}
Аргументы функции \texttt{get\_expmt}: \texttt{T} -- период функции.
\newline


\begin{lstlisting}[style=python_white, caption=Функция для вычисления коэффициентов Фурье, label=lst:fourier]
def fourier(func, t0, T, N):
    sinmt, cosmt = get_sincosmt(T)
    a = []
    b = []
    for i in range(N + 1):
        sin = lambda x: sinmt([i, x])
        cos = lambda x: cosmt([i, x])

        a.append(dot_product(func, cos, t0, t0 + T) * 2 / T)
        b.append(dot_product(func, sin, t0, t0 + T) * 2 / T)
    return a, b
\end{lstlisting}
Аргументы функции \texttt{fourier}: \texttt{func} -- функция, для которой вычисляются коэффициенты, \texttt{t0} -- начало промежутка, на котором проводится разложение, \texttt{T} -- период функции, \texttt{N} -- количество коэффициентов.
\newline

\begin{lstlisting}[style=python_white, caption=Функция для вычисления коэффициентов Фурье, label=lst:fourier_exp]
def fourier_exp(func, t0, T, N):
    expmt = get_expmt(T)
    c = []
    for i in range(-N, N + 1):
        exp = lambda x: expmt([-i, x])
        c.append(dot_product(func, exp, t0, t0 + T) / T)

    return c
\end{lstlisting}
Аргументы функции \texttt{fourier\_exp}: \texttt{func} -- функция, для которой вычисляются коэффициенты, \texttt{t0} -- начало промежутка, на котором проводится разложение, \texttt{T} -- период функции, \texttt{N} -- количество коэффициентов.
\newline

\begin{lstlisting}[style=python_white, caption=Получение функции частичной суммы рядя Фурье до $N$, label=lst:fourier_func]
def fourier_func(a, b, T, N):
    sinmt, cosmt = get_sincosmt(T)
    return lambda x: a[0] / 2 + sum([a[i] * cosmt([i, x]) + b[i] * sinmt([i, x]) for i in range(1, N + 1)])
\end{lstlisting}
Аргументы функции \texttt{fourier\_func}: \texttt{a} и \texttt{b} -- коэффициенты Фурье, \texttt{T} -- период функции, \texttt{N} -- количество коэффициентов.
\newline

\begin{lstlisting}[style=python_white, caption=Получение функции частичной суммы рядя Фурье до $N$, label=lst:fourier_func_exp]
def fourier_exp_func(c, T, N):
    expmt = get_expmt(T)
    return lambda x: sum([c[i + len(c) // 2] * expmt([i, x]) for i in range(-N, N + 1)])
\end{lstlisting}
Аргументы функции \texttt{fourier\_exp\_func}: \texttt{c} -- коэффициенты Фурье, \texttt{T} -- период функции, \texttt{N} -- количество коэффициентов.
\newline
    
\begin{lstlisting}[style=python_white, caption=Функция для получения разложения Фурье функции, label=lst:fourierise]
def fourierise(func, t0, T, N):
    a, b = fourier(func, t0, T, N)
    return fourier_func(a, b, T, N)
\end{lstlisting}
Аргументы функции \texttt{fourierise}: \texttt{func} -- функция, для которой проводится разложение, \texttt{t0} -- начало промежутка, на котором проводится разложение, \texttt{T} -- период функции, \texttt{N} -- количество коэффициентов.
\newline

\begin{lstlisting}[style=python_white, caption=Функция для получения разложения Фурье функции, label=lst:fourierise_exp]
def fourierise_exp(func, t0, T, N):
    c = fourier_exp(func, t0, T, N)
    return fourier_exp_func(c, T, N)
\end{lstlisting}
Аргументы функции \texttt{fourierise\_exp}: \texttt{func} -- функция, для которой проводится разложение, \texttt{t0} -- начало промежутка, на котором проводится разложение, \texttt{T} -- период функции, \texttt{N} -- количество коэффициентов.
\newline

\begin{lstlisting}[style=python_white, caption=Функция для отрисовки графиков, label=lst:plot_func]
def plot_func(func, fourier_func, t0, T, caption = '', title = ''):
    # limits
    x_min = t0 - T * 0.5
    x_max = t0 + 1.5 * T
    ymin = min(func(np.linspace(x_min, x_max, 100))) 
    ymax = max(func(np.linspace(x_min, x_max, 100))) 

    ymax = ymax + 0.1 * (ymax - ymin)
    ymin = ymin - 0.1 * (ymax - ymin)

    x = np.linspace(x_min, x_max, 1000)
    plt.figure(figsize=(8, 5)) 

    plt.ylim(ymin, ymax)
    plt.plot(x, func(x))
    if fourier_func != func:
        plt.plot(x, fourier_func(x))
        plt.plot([t0, t0], [ymin, ymax], 'g--')
        plt.plot([t0 + T, t0 + T], [ymin, ymax], 'g--')
    # add labels
    plt.xlabel('t')
    plt.ylabel('f(t)')
    if fourier_func != func:
        plt.legend(['Source function', 'Fourier function'], loc='upper right')
    else:
        plt.legend(['Source function'], loc='upper right')
    # add caption
    plt.title(caption)
    plt.grid()
    plt.savefig(title + '.png')
\end{lstlisting}
Аргументы функции \texttt{plot\_func}: \texttt{func} -- исходная функция, \texttt{fourier\_func} -- функция частичной суммы ряда Фурье, \texttt{t0} -- начало промежутка, на котором проводится разложение, \texttt{T} -- период функции, \texttt{caption} -- подпись к графику, \texttt{title} -- название файла для сохранения.
\newline

\begin{lstlisting}[style=python_white, caption=Функция для отрисовки графиков (параметрическая функция), label=lst:plot_parametric_func]
def plot_parametric_func(func, fourier_func, t0, T, caption = '', title = ''):
    t = np.linspace(t0, t0 + T, 1000)
    plt.figure(figsize=(8, 5))
    plt.plot(func(t).real, func(t).imag)
    if fourier_func != func:
        plt.plot(fourier_func(t).real, fourier_func(t).imag)
    # add labels
    plt.xlabel('Re')
    plt.ylabel('Im')
    if fourier_func != func:
        plt.legend(['Source function', 'Fourier function'], loc='upper right')
    else:
        plt.legend(['Source function'], loc='upper right')
    # add caption
    plt.title(caption)
    plt.grid()
    plt.savefig(title + '.png') 
\end{lstlisting}
Аргументы функции \texttt{plot\_parametric\_func}: \texttt{func} -- исходная функция, \texttt{fourier\_func} -- функция частичной суммы ряда Фурье, \texttt{t0} -- начало промежутка, на котором проводится разложение, \texttt{T} -- период функции, \texttt{caption} -- подпись к графику, \texttt{title} -- название файла для сохранения.
\newline

\begin{lstlisting}[style=python_white, caption=Функция для разложения в ряд Фурье с различными значениями $N$, label=lst:calc_and_plot]
def calc_and_plot(func, t0, T, N, title):
    plot_func(func, func, t0, T, 'Source function', title)
    for n in N:
        right_caption = f'Fourier function, N = {n}'
        fourier_func = fourierise(func, t0, T, n)
        
        plot_func(func, fourier_func, t0, T, right_caption, title + f'_N_{n}')
\end{lstlisting}
Аргументы функции \texttt{calc\_and\_plot}: \texttt{func} -- исходная функция, \texttt{t0} -- начало промежутка, на котором проводится разложение, \texttt{T} -- период функции, \texttt{N} -- количество коэффициентов $N$, \texttt{title} -- название файла для сохранения.
\newline

\begin{lstlisting}[style=python_white, caption=Функция для разложения в ряд Фурье с различными значениями $N$, label=lst:calc_and_plot_exp]
def calc_and_plot_exp(func, t0, T, N, title):   
    plot_func(func, func, t0, T, 'Source function', title)
    for n in N:
        right_caption = f'Fourier function (exp), N = {n}'
        fourier_func = fourierise_exp(func, t0, T, n)
        
        plot_func(func, fourier_func, t0, T, right_caption, title + f'_N_{n}')
\end{lstlisting}
Аргументы функции \texttt{calc\_and\_plot\_exp}: \texttt{func} -- исходная функция, \texttt{t0} -- начало промежутка, на котором проводится разложение, \texttt{T} -- период функции, \texttt{N} -- количество коэффициентов $N$, \texttt{title} -- название файла для сохранения.
\newline
\begin{lstlisting}[style=python_white, caption=Функция для разложения в ряд Фурье с различными значениями $N$ (параметрическая функция), label=lst:calc_and_plot_exp]
def calc_and_plot_parametric(func, t0, T, N, title):
    plot_parametric_func(func, func, t0, T, 'Source function', title)
    for n in N:
        right_caption = f'Fourier function, N = {n}'
        fourier_func = fourierise_exp(func, t0, T, n)
        
        plot_parametric_func(func, fourier_func, t0, T, right_caption, title + f'_N_{n}')
\end{lstlisting}
Аргументы функции \texttt{calc\_and\_plot\_parametric}: \texttt{func} -- исходная функция, \texttt{t0} -- начало промежутка, на котором проводится разложение, \texttt{T} -- период функции, \texttt{N} -- количество коэффициентов $N$, \texttt{title} -- название файла для сохранения.
\newline

\begin{lstlisting}[style=python_white, caption=Функция для вывода коэффициентов, label=lst:print_fourier_coefficients]
def print_fourier_coefficients(a, b):
    for i in range(len(a)):
        print(f'a_{i} = \t{a[i]:.5f}, \tb_{i} = \t{b[i]:.5f}')
\end{lstlisting}
Аргументы функции \texttt{print\_fourier\_coefficients}: \texttt{a} и \texttt{b} -- коэффициенты Фурье.
\newline

\begin{lstlisting}[style=python_white, caption=Функция для вывода коэффициентов, label=lst:print_fourier_coefficients_exp]
def print_fourier_exp_coefficients(c):
    for i in range(len(c)):
        print(f'c_{i - len(c) // 2} = \t{c[i]:.5f}')
\end{lstlisting}
Аргументы функции \texttt{print\_fourier\_coefficients\_exp}: \texttt{c}-- коэффициенты Фурье.
\newline

\begin{lstlisting}[style=python_white, caption=Функция для проверки равенства Персеваля, label=lst:perseval_check]
def perseval_check(func, N):
    norm_squared = dot_product(func, func, -np.pi, np.pi)
    a, b = fourier(func, -np.pi, 2 * np.pi, N)
    c = fourier_exp(func, -np.pi, 2 * np.pi, N)
    c_sum = 2 * np.pi * sum(abs(c[i]) ** 2 for i in range(len(c)))
    ab_sum = np.pi * (a[0] ** 2 / 2 + sum([a[i] ** 2 + b[i] ** 2 for i in range(1, N + 1)]))
    print(f'Norm squared: {norm_squared:.5f}, \tSum of |c_i|^2: {c_sum:.5f}, \tSum of |a_i|^2 + |b_i|^2: {ab_sum:.5f}')
\end{lstlisting}
Аргументы функции \texttt{perseval\_check}: \texttt{func} -- функция, для которой проверяется равенство Персеваля, \texttt{N} -- количество коэффициентов $N$.
\newline


\begin{lstlisting}[style=python_white, caption=Пример работы с функциями, label=lst:example1, belowskip=-0.8\baselineskip]
func = np.vectorize(lambda x: 1 if 0 <= (x - 1) % 3 < 1 else 2)
print("Func 1")
a, b = fourier(func, 1, 3, 3)
print_fourier_coefficients(a, b)
c = fourier_exp(func, 1, 3, 3)
print_fourier_exp_coefficients(c)

calc_and_plot(func, 1, 3, [1, 2, 5, 15, 30], './media/plots/func_1')
calc_and_plot_exp(func, 1, 3, [1, 2, 5, 15, 30], './media/plots/func_1_exp')
\end{lstlisting}
В данном примере функция \texttt{func} (согласно уравнению (\ref{eq:func_1})) -- функция с периодом $T = 3$. 
\texttt{a, b, c} -- коэффициенты Фурье для $N = 3$
\newline

